\documentclass{article}

\usepackage{listings}


\begin{document}

\title{Abstracts (encoding and decoding of text files){}}
\author{Sam Jongman, Huub Visser}

\maketitle

\begin{abstract}
We have made a program that encodes a text file per line into numbers, whilst keeping special character intact. 
\end{abstract}

\section{Encoding}
After the user has chosen a file, they have the option to process it and save the encoded file as a text file and save the index file. The index file is a CSV-file in which the translation (corresponding letters to the words), frequency of the words within the whole file and the amount of times a word appears in a line (abstract) is written. When it encodes a file it does it by reading the text file line by line. Each different word gets a different number assigned to it so if a word appears multiple times in a file it has the same number assigned to it. It also creates a textual histogram of the amount of times a word begins with a certain letter.

\section{Decoding}
Unfortunately, we did not have enough time to program a decoder and it also doesn’t calculate the TF/IDF-scores of a line. 

\section{Source code listing}
Below you will find the listing of our program. We have tried to use the suggestions in the assigneme

\begin{lstlisting}[language=Python]
print('''
Student .......: Sam Jongman, Huub Visser
Number ........: s2550040, s2568861
Assignment ....: Abstracts
Last Edit Date : Nov, 7 2019

Description ...: This program encodes and decodes text files that the user can upload
''')

import os
import math

def main_menu(selected_file):
                                                    # This asks what the user wants to do
    clear_console();
    options = "1"
    print('Welcome to Abstracts')
    print('')
    print('This program is used to encode or decode text files.')
    print('')
    if encode_or_decode(selected_file) == 'encode':
        print('Selected file: ' + selected_file)
        print('Output file: ' + get_output_file(selected_file))
        print('Index file: ' + get_index_file(selected_file))
    elif encode_or_decode(selected_file) == 'decode':
        print('Selected file: ' + selected_file)
        print('Output file: ' + get_output_file(selected_file))
    else:
        print('No .txt or index.csv file selected.')
    print('')
    print('1: Select a file')
    if selected_file.lower().endswith('txt'): #This checks what kind of file it is, if it isn't a text file or csv file it asks agian
        print('2: Encode ' + selected_file)
        options += ", 2" 
    if selected_file.lower().endswith('csv'):
        print('2: Decode ' + selected_file)
        options += ", 2" 
    print('0: Exit program')
    print('')
    return int(input('Please type ' + options + ' or 0: '))

def get_output_file(file):
    if encode_or_decode(file) == "encode":
        return file + '.encoded.txt'
    return file.replace('.index.csv', '.decoded.txt') #this creates a filename for a save file

def get_index_file(file):
    if encode_or_decode(file) == "encode":   #this creates a filename for a save file
        return file + '.index.csv'

def encode_or_decode(selected_file):
    if selected_file.lower().endswith('txt'): #this checks if a the selected file is a txt or cvs file
        return 'encode' 
    if selected_file.lower().endswith('index.csv'):
        return 'decode' 
    return 'wrong file' 

def clear_console():
    print("\n" * 100) #prints some white lines

def file_menu():
    clear_console();
    print('')
    print('Select a .txt file to encode or a .index.csv file to decode')
    print('')
    print('The current directory is ' + os.getcwd()) # checks current directory
    print('')
    menu = 1
    print('1: ../')
    files = {0:'', 1: '../'}
    for entry in os.listdir(os.getcwd()):
        if not entry.startswith('.'):
            menu = menu + 1
            if os.path.isdir(entry):
                entry = entry + '/'
            print(str(menu) + ': ' + entry)
            files[menu] = entry
    print('0: Exit program')
    print('')
    choice = int(input('Please type 1 to ' + str(menu) + ' or 0: ')) # with this function you can navigate through your direcory, much easier then typing it in yourself
    return files[choice]

def select_file():
    file_name = file_menu()
    if file_name.endswith('/'): #if it is another directory it goes in there otherwise it opens the file
        os.chdir(file_name)
        return select_file()
    if len(file_name) == 0:
        return ''
    return file_name

def count_word_in_abstracts(word, abstr_freqs): #how many words are in an abstract?
    count = 0
    for abstract_index in abstr_freqs:
        if word in abstr_freqs[abstract_index]:
            count += 1
    return count

def write_decoded_files(selected_file, words, codes, abstr_freqs,
                    total_freqs, abstracts, encodedAbstracts): #unfortunatly we did not have time to make a decoder...
    print("to do")
    
def write_encoded_files(selected_file, words, codes, abstr_freqs, #this writes the encoded txt file
                    total_freqs, abstracts, encodedAbstracts):
    output_file = get_output_file(selected_file)
    file = open(output_file, "w+")
    for abstract_index in encodedAbstracts:
        file.write(encodedAbstracts[abstract_index] + '\n')
    file.close()

    csv_file = get_index_file(selected_file) #this writes the csv file
    file = open(csv_file, "w+")
    file.write("word,number,frequency,abstracts\n") #title of csv file

    for word in codes:
        if codes[word] == 0:
            continue
        file.write(word + ',')
        file.write(str(codes[word]) + ',')
        if word in total_freqs:
            file.write(str(total_freqs[word]) + ',')
            file.write(str(count_word_in_abstracts(word, abstr_freqs)) + '\n')
        else:
            file.write("0,")
            file.write("0\n")

    for word in codes:
        if codes[word] != 0:
            continue
        file.write(word + ',')
        file.write(str(codes[word]) + ',')
        if word in total_freqs:
            file.write(str(total_freqs[word]) + ',')
            file.write(str(count_word_in_abstracts(word, abstr_freqs)) + '\n')
        else:
            file.write("0,")
            file.write("0\n")

    file.close()

def print_encode_result(selected_file, words, codes, total_freqs, abstracts):
    orig_size = os.path.getsize(selected_file);
    output_size = os.path.getsize(get_output_file(selected_file));
    
    index_size = 0
    for word in codes:
        index_size += len(word)
    
    comp_rate = math.floor((output_size + index_size) / orig_size * 100) #comperssion rate calculator
    print()
    print("---------------------------")
    print("Found " + str(len(words)) + " unique words")         #this prints the amount of unique words found in how many abstracts
    print("in " + str(len(abstracts)) + " abstracts.")
    print("The compression rate is " + str(comp_rate) + "%")
    print("---------------------------")
    print()

    words_per_letter = {}
    unique_words_per_letter = {}                        # this creates dicts for the letters per word and the amount of unique letters per word
    for letter in "abcdefghijklmnopqrstuvwxyz":
        words_per_letter[letter] = 0
        unique_words_per_letter[letter] = 0
        for word in total_freqs:
            if word[:1] == letter:
                if word.lower() == word:
                    words_per_letter[letter] += total_freqs[word]
                    unique_words_per_letter[letter] += 1 

    all_players = {}
    for letter in "abcdefghijklmnopqrstuvwxyz":
        all_players[letter] = letter
        
    weener_list = {}
    while len(weener_list) < 26:
        weener = next(iter(all_players))
        weener_freq = words_per_letter[weener]
        for letter in all_players:
            if letter == weener:
                continue
            if words_per_letter[letter] > weener_freq:
                weener = letter
                weener_freq = words_per_letter[letter]
        del all_players[weener]
        weener_list[weener] = weener_freq        
    for letter in weener_list:
        most_words = words_per_letter[max(words_per_letter, key=words_per_letter.get)]
        bars_equal = int(round(words_per_letter[letter] / most_words * 40))
        bars_plus = int(round(unique_words_per_letter[letter] / most_words * 40))               #this calculates how many + and = has to be printed in the textual histogram
        amount_bars_equal = ''
        amount_bars_plus = ''
        count_equal = 0
        count_plus = 0
        while bars_plus > count_plus:
            amount_bars_plus = amount_bars_plus + '+'
            count_plus = count_plus + 1
        while bars_equal - count_plus > count_equal :
            amount_bars_equal = amount_bars_equal + '='
            count_equal = count_equal + 1
        
        fmt = '{: <40}'.format(amount_bars_plus + amount_bars_equal)
        print(letter + ' | '  + fmt, end='')
        fmt = '{:>5}'.format(unique_words_per_letter[letter])                   # this makes the histogram orginized and places spaces if there are no + or = to place
        print(' | ' + fmt , end='')
        fmt = '{:>5}'.format(words_per_letter[letter])
        print(' | ' + fmt + ' | ' )
        
        
def main(selected_file):
    words, codes, abstr_freqs, total_freqs = {}, {}, {}, {}
    abstracts, encodedAbstracts = {}, {}
    
    while True:
        choice = main_menu(selected_file)                           

        if choice == 1:
            selected_file = select_file()
            if len(selected_file) == 0:
                print('')
                print('Goodbye!')
                return
     
        elif (choice == 2) and encode_or_decode(selected_file) == "encode":
            encode(selected_file, words, codes, abstr_freqs,
                           total_freqs, abstracts, encodedAbstracts)
            write_encoded_files(selected_file, words, codes, abstr_freqs,
                        total_freqs, abstracts, encodedAbstracts)
            print("")
            print_encode_result(selected_file, words, codes, 
                                total_freqs, abstracts)
            print()
            print('Thank you, come again.')
            return
     
        elif (choice == 2) and encode_or_decode(selected_file) == "decode" :
            decode(selected_file, words, codes, abstr_freqs,
                           total_freqs, abstracts, encodedAbstracts)
            write_decoded_files(selected_file, words, codes, abstr_freqs,
                        total_freqs, abstracts, encodedAbstracts)
            print()
            print('Thank you, come again.')
            return
    
        elif choice == 0:
            print('')
            print('Goodbye!')
            return
    
def store_word(word, abstract_nr, abstr_freqs, total_freqs, codes, words): #this stores a word in the coded file, if a word isn't seen yet is creates a new key and value and if it is already seen it adds one to the value

    if not word in codes:
        wordCount = len(words) + 1
        words[wordCount] = word
        codes[word] = wordCount

    myWord = word.lower()                                   
    
    if not myWord in total_freqs:
        total_freqs[myWord] = 1
    else:
        total_freqs[myWord] += 1
    
    if not abstract_nr in abstr_freqs:
        abstr_freqs[abstract_nr] = {}
    
    if not myWord in abstr_freqs[abstract_nr]:
        abstr_freqs[abstract_nr][myWord] = 1
    else:
        abstr_freqs[abstract_nr][myWord] += 1

        
def decode(file_name, words, codes,  abstr_freqs, total_freqs,
           abstracts, encodedAbstracts):
    print("to do")
        
def encode(file_name, words, codes,  abstr_freqs, total_freqs,
           abstracts, encodedAbstracts):
    word = ''
    abstractCount =  0 
    #This is will check if a character is a \ or a number, which it will have to encode in a different way
    file = open(file_name)
    for abstract in file.readlines():
        coded=''
        abstracts[abstractCount] = abstract;
        for char in abstract:
            if char.isalpha():
                word += char
                continue
            if char.isnumeric():
                coded += '\\' + char
                continue
            if char == '\\':
                coded += '\\' + char
                continue
            if word == '':
                coded += char
                continue
            
            store_word(word, abstractCount, abstr_freqs, total_freqs,
                       codes, words)
            coded += str(codes[word]) + char
            word =  ''

        if word != '': 
            store_word(word, abstractCount, abstr_freqs, total_freqs,
                       codes, words)
            coded += str(codes[word])

        encodedAbstracts[abstractCount] = coded
        coded=''
        abstractCount += 1
    
   
    for index in words:                                         # Put all cap words without lowercase in codes with code 0
        word = words[index]
        low_word = word.lower()
        if low_word not in codes:
            codes[low_word] = 0


main('')

\end{lstlisting}
\end{document}
